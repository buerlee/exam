%!TEX program = xelatex

\documentclass[12pt,a4paper]{ctexart}
%\usepackage[T1]{fontenc}
\usepackage{geometry}
\geometry{verbose,tmargin=2.5cm,bmargin=2.5cm,lmargin=2.5cm,rmargin=2.5cm}
\usepackage{array}
\usepackage{lastpage} % 获取总页数
\usepackage{fancyhdr} % 用于定制页眉页脚
\newcommand{\myDash}{--\,--\,--\,--\,--\,--\,--\,--} % 虚线

\usepackage{pstricks} % 用于画方框
\definecolor{myGrey}{rgb}{0.8,0.8,0.8} % 定义灰色

\setlength{\baselineskip}{1.2\baselineskip} % 正文1.2倍行距
\renewcommand{\arraystretch}{1.2} % 表格1.2倍行距

%===页眉页脚定制===%

%---首页页眉页脚---%
\fancypagestyle{plain}{
\fancyhead{}
\fancyfoot[C]{{\footnotesize 第\thepage 页,共\pageref{LastPage} 页}}
\renewcommand{\headrulewidth}{0pt}
}

%---其他页页眉页脚---
\pagestyle{fancy}
\fancyhead[C]{{\footnotesize {\kaishu 中山大学本科生期末考试试卷}}}
\fancyfoot[C]{{\footnotesize 第\thepage 页,共\pageref{LastPage} 页}}
\renewcommand{\headrulewidth}{0pt}

\date{}

%===试卷头===%

\begin{document}
\thispagestyle{plain}

\begin{center}
\textbf{{\zihao{2} 中山大学本科生期末考试}}
\end{center}

\smallskip

\begin{center}
{\zihao{3} 考试科目:《\#\#\#\#\#》(A卷/B卷)}
\end{center}

{\kaishu
\noindent\begin{tabular}{>{\raggedright}m{0.5\textwidth}>{\raggedright}m{0.5\textwidth}}
学年学期:\#\#\#\#学年第\#学期 & 姓\hspace{2em}名:\underline{\hspace{12em}}\tabularnewline
学\hspace{0.5em}院/系:\#\#\# & 学\hspace{2em}号:\underline{\hspace{12em}}\tabularnewline
考试方式:闭卷/开卷 & 年级专业:\underline{\hspace{12em}}\tabularnewline
考试时长:\#分钟 & 班\hspace{2em}别:\underline{\hspace{12em}}\tabularnewline
\end{tabular}}

\bigskip{}

\noindent\psscalebox{1.2 1.2} % Change this value to rescale the drawing.
{
\begin{pspicture}(0,-0.1)(1.2,0.4)
\psframe[linecolor=black, linewidth=0.02, fillstyle=solid,fillcolor=myGrey, dimen=outer, framearc=0.3](1.2,0.4)(0.0,-0.4)
\rput[c](0.6,0.0){{\heiti 警示}}
\end{pspicture}
}{\fangsong《中山大学授予学士学位工作细则》第八条:“考试作弊者,不授予学士学位。”}

\smallskip{}

\begin{center}
\myDash\colorbox{myGrey}{\kaishu{\zihao{5} 以下为试题区域,共\#道题,总分100分,考生请在答题纸上作答}}\myDash
\end{center}


%===试题部分===%




\end{document}
